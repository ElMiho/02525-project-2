\documentclass{article}
\usepackage[utf8]{inputenc}
\usepackage{amsmath, amssymb}
\usepackage{mathtools}
\usepackage{a4wide}
\usepackage{appendix}
\usepackage{listings}
\usepackage{float}
\usepackage{subcaption}
\usepackage{hyperref}

\title{Salami}
\author{Hugo M. Nielsen (s214734) \and Mikael H. Hoffmann (s214753) \and Christian Valentin Kjær (s211469) \and Jacob Tuxen (s194572)}
\date{\today}

\begin{document}
\maketitle
\section{Problem and background}
When producing sausages and salami it is of great interest to determine the meat/fat ratio. Manually annotating sausages and salami requires a food expert and this is very time and resource consuming. 
Therefore we will look at this problem by using image analysis and investigate whether it is possible to create a model which can 
 
\section{Data and experiments}


\section{Mathematical model}

\subsection{Multivariate model}
Every pixel in the multi-spectral images corresponds to meat or fat. Assuming that each pixel is an independent multivariate normal random variable, with distribution $\mathcal{N}(\boldsymbol{\mu}_i,\boldsymbol{\Sigma}),\:i\in\{1,2\}$, corresponding to meat or fat respectively. For a given pixel $\textbf{x}\in\mathbb{R}^n$ the mathematical problem is to determine whether $\textbf{x}\sim\mathcal{N}(\boldsymbol{\mu}_1,\boldsymbol{\Sigma})$ or $\textbf{x}\sim\mathcal{N}(\boldsymbol{\mu}_2,\boldsymbol{\Sigma})$.\\
An observation $\textbf{x}$ is most likely to be in class 1 if $P(C_1|\textbf{x})>P(C_2|\textbf{x})$ by Bayes formula i.e $P(\textbf{x}|C_1)P(C_1)>P(\textbf{x}|C_2)P(C_2)$. Due to the assumption of each class being multivariate normally distributed with prior probabilities $P(C_i)=p_i,\:i\in\{1,2\}$, these distributions are known. By applying the logarithm and removing any terms not dependent on the class we define the threshold $S_i(\textbf{x})$ by,
\begin{align*}
    P(\textbf{x}|C_i)p_1&=\frac{1}{\sqrt{2\pi}^n\sqrt{|\boldsymbol{\Sigma}}|}\text{exp}\bigg[-\frac{1}{2}(\textbf{x}-\boldsymbol{\mu_1})^T\boldsymbol{\Sigma}^{-1}(\textbf{x}-\boldsymbol{\mu_i})\bigg]p_i \\
    S_i(\textbf{x})&=-\frac{1}{2}(\textbf{x}-\boldsymbol{\mu_i})^T\boldsymbol{\Sigma}^{-1}(\textbf{x}-\boldsymbol{\mu}_i)+\log(p_i) \\
    &=-\frac{1}{2}\big[\textbf{x}^T\boldsymbol{\Sigma}^{-1}-\boldsymbol{\mu}^T_i\boldsymbol{\Sigma}^{-1}(\textbf{x}-\boldsymbol{\mu}_i)]+\log(p_i)\\
    &=-\frac{1}{2}\big[\textbf{x}^T\boldsymbol{\Sigma}^{-1}\textbf{x}-\textbf{x}^T\boldsymbol{\Sigma}^{-1}\boldsymbol{\mu}_i-\boldsymbol{\mu}_i^T\boldsymbol{\Sigma}^{-1}\textbf{x}+\boldsymbol{\mu}_i^T\boldsymbol{\Sigma}^{-1}\boldsymbol{\mu}_i\big]+\log(p_i) \\
    &=\frac{1}{2}\big[\textbf{x}^T\boldsymbol{\Sigma}^{-1}\boldsymbol{\mu}_i+\boldsymbol{\mu}_i^T\boldsymbol{\Sigma}^{-1}\textbf{x}-\boldsymbol{\mu}_i^T\boldsymbol{\Sigma}^{-1}\boldsymbol{\mu}_i\big]+\log(p_i).
\end{align*}
Since $\boldsymbol{\Sigma}$, is symmetric we obtain  
\begin{align*}
    S_(\textbf{x})&=\frac{1}{2}\big[\textbf{x}^T\boldsymbol{\Sigma}^{-1}\boldsymbol{\mu}_i+\boldsymbol{\mu}_i^T\boldsymbol{\Sigma}^{-1}\textbf{x}-\boldsymbol{\mu}_i^T\boldsymbol{\Sigma}^{-1}\boldsymbol{\mu}_i\big]+\log(p_i)\\
    &=\textbf{x}^T\boldsymbol{\Sigma}^{-1}\boldsymbol{\mu}_i-\frac{1}{2}\boldsymbol{\mu}_i^T\boldsymbol{\Sigma}^{-1}\boldsymbol{\mu}_i+\log(p_i).\\
\end{align*}
Using this threshold the classification for each pixel can be made from the following function.
\begin{align*}
    \tau(\textbf{x})=\begin{cases}
        C_1 \quad & S_1(\textbf{x})\geq S_2(\textbf{x}) \\
        C_2 \quad & S_2(\textbf{x})<S_1(\textbf{x}).
    \end{cases}
\end{align*}
\section{Results}
\begin{tabular}{|c|c|c|c|c|c|c}
\hline
Test day & 01 & 06 & 13 & 20 & 28 \\
\hline
 Training day: 01 & - & 0.0678663 & 0.052972  & 0.0992975 & 0.146353  \\
 Training day: 06 & 0.0245613 & - & 0.0209617 & 0.0335357 & 0.033407  \\
 Training day: 13 & 0.0421846 & 0.0585285 & - & 0.0513861 & 0.0637519 \\
 Training day: 20 & 0.0347886 & 0.0550122 & 0.0346372 & - & 0.0394442 \\
 Training day: 28 & 0.0399817 & 0.0619729 & 0.0358711 & 0.0491302 & - \\
\hline
\end{tabular}
\section{Conclude}


\section{References}


\end{document}